\documentclass[10pt, letterpaper]{article}

% Packages:
\usepackage[
    ignoreheadfoot, % set margins without considering header and footer
    top=2 cm, % seperation between body and page edge from the top
    bottom=2 cm, % seperation between body and page edge from the bottom
    left=2 cm, % seperation between body and page edge from the left
    right=2 cm, % seperation between body and page edge from the right
    footskip=1.0 cm, % seperation between body and footer
    % showframe % for debugging 
]{geometry} % for adjusting page geometry
\usepackage{titlesec} % for customizing section titles
\usepackage{tabularx} % for making tables with fixed width columns
\usepackage{array} % tabularx requires this
\usepackage[dvipsnames]{xcolor} % for coloring text
\definecolor{primaryColor}{RGB}{0, 0, 0} % define primary color
\usepackage{enumitem} % for customizing lists
\usepackage{fontawesome5} % for using icons
\usepackage{amsmath} % for math
\usepackage[
    pdftitle={Charan N's CV},
    pdfauthor={Charan N},
    pdfcreator={LaTeX with RenderCV},
    colorlinks=true,
    urlcolor=primaryColor
]{hyperref} % for links, metadata and bookmarks
\usepackage[pscoord]{eso-pic} % for floating text on the page
\usepackage{calc} % for calculating lengths
\usepackage{bookmark} % for bookmarks
\usepackage{lastpage} % for getting the total number of pages
\usepackage{changepage} % for one column entries (adjustwidth environment)
\usepackage{paracol} % for two and three column entries
\usepackage{ifthen} % for conditional statements
\usepackage{needspace} % for avoiding page brake right after the section title
\usepackage{iftex} % check if engine is pdflatex, xetex or luatex

% Ensure that generate pdf is machine readable/ATS parsable:
\ifPDFTeX
    \input{glyphtounicode}
    \pdfgentounicode=1
    \usepackage[T1]{fontenc}
    \usepackage[utf8]{inputenc}
    \usepackage{lmodern}
\fi

\usepackage{charter}

% Some settings:
\raggedright
\AtBeginEnvironment{adjustwidth}{\partopsep0pt} % remove space before adjustwidth environment
\pagestyle{empty} % no header or footer
\setcounter{secnumdepth}{0} % no section numbering
\setlength{\parindent}{0pt} % no indentation
\setlength{\topskip}{0pt} % no top skip
\setlength{\columnsep}{0.15cm} % set column seperation
\pagenumbering{gobble} % no page numbering

\titleformat{\section}{\needspace{4\baselineskip}\bfseries\large}{}{0pt}{}[\vspace{1pt}\titlerule]

\titlespacing{\section}{
    % left space:
    -1pt
}{
    % top space:
    0.3 cm
}{
    % bottom space:
    0.2 cm
} % section title spacing

\renewcommand\labelitemi{$\vcenter{\hbox{\small$\bullet$}}$} % custom bullet points
\newenvironment{highlights}{
    \begin{itemize}[
        topsep=0.10 cm,
        parsep=0.10 cm,
        partopsep=0pt,
        itemsep=0pt,
        leftmargin=0 cm + 10pt
    ]
}{
    \end{itemize}
} % new environment for highlights


\newenvironment{highlightsforbulletentries}{
    \begin{itemize}[
        topsep=0.10 cm,
        parsep=0.10 cm,
        partopsep=0pt,
        itemsep=0pt,
        leftmargin=10pt
    ]
}{
    \end{itemize}
} % new environment for highlights for bullet entries

\newenvironment{onecolentry}{
    \begin{adjustwidth}{
        0 cm + 0.00001 cm
    }{
        0 cm + 0.00001 cm
    }
}{
    \end{adjustwidth}
} % new environment for one column entries

\newenvironment{twocolentry}[2][]{
    \onecolentry
    \def\secondColumn{#2}
    \setcolumnwidth{\fill, 4.5 cm}
    \begin{paracol}{2}
}{
    \switchcolumn \raggedleft \secondColumn
    \end{paracol}
    \endonecolentry
} % new environment for two column entries

\newenvironment{threecolentry}[3][]{
    \onecolentry
    \def\thirdColumn{#3}
    \setcolumnwidth{, \fill, 4.5 cm}
    \begin{paracol}{3}
    {\raggedright #2} \switchcolumn
}{
    \switchcolumn \raggedleft \thirdColumn
    \end{paracol}
    \endonecolentry
} % new environment for three column entries

\newenvironment{header}{
    \setlength{\topsep}{0pt}\par\kern\topsep\centering\linespread{1.5}
}{
    \par\kern\topsep
} % new environment for the header

\newcommand{\placelastupdatedtext}{% \placetextbox{<horizontal pos>}{<vertical pos>}{<stuff>}
  \AddToShipoutPictureFG*{% Add <stuff> to current page foreground
    \put(
        \LenToUnit{\paperwidth-2 cm-0 cm+0.05cm},
        \LenToUnit{\paperheight-1.0 cm}
    ){\vtop{{\null}\makebox[0pt][c]{
        \small\color{gray}\textit{Last updated in September 2024}\hspace{\widthof{Last updated in September 2024}}
    }}}%
  }%
}%

% save the original href command in a new command:
\let\hrefWithoutArrow\href

% new command for external links:


\begin{document}
    \newcommand{\AND}{\unskip
        \cleaders\copy\ANDbox\hskip\wd\ANDbox
        \ignorespaces
    }
    \newsavebox\ANDbox
    \sbox\ANDbox{$|$}

    \begin{header}
        \fontsize{25 pt}{25 pt}\selectfont CHARAN N

        \vspace{5 pt}

        \normalsize
        \mbox{Bengaluru India}%
        \kern 5.0 pt%
        \AND%
        \kern 5.0 pt%
        \mbox{\hrefWithoutArrow{mailto:charann.cd23@rvce.edu.in}{charann.cd23@rvce.edu.in}}%
        \kern 5.0 pt%
        \AND%
        \kern 5.0 pt%
        \mbox{\hrefWithoutArrow{tel:+91-900-821-39-05}{9008213905}}%
        \kern 5.0 pt%
        \AND%
        \kern 5.0 pt%
        \mbox{\hrefWithoutArrow{https://linkedin.com/in/charan-nagaraj}{linkedin.com/in/charan-nagaraj}}%
        \kern 5.0 pt%
        \AND%
        \kern 5.0 pt%
        \mbox{\hrefWithoutArrow{https://github.com/charan-sharan}{github.com/charan-sharan}}%
    \end{header}

    \vspace{5 pt - 0.3 cm}

    \section{Education}
        
        \begin{twocolentry}{
            Sep 2023 – Present 
        }
            \textbf{R.V. College of Engineering}, Bachelor of Engineering in Computer Science\end{twocolentry}

        \vspace{0.10 cm}
        \begin{onecolentry}
            \begin{highlights}
                \item CGPA: 9.41/10 
            \end{highlights}
        \end{onecolentry}

    \section{Technologies}

        \begin{onecolentry}
            \textbf{Languages:} C++, C, Python, Bash Scripting \end{onecolentry}

        \vspace{0.2 cm}

        \begin{onecolentry}
            \textbf{DevOps \& CI/CD Tools:} Docker, Kubernetes, GitHub Actions, Ansible, Terraform \end{onecolentry}

        \vspace{0.2 cm}
        
        \begin{onecolentry}
            \textbf{Development Boards:} Raspberry Pi, ESP32, Arduino \end{onecolentry}

        \vspace{0.2 cm}

        \begin{onecolentry}
            \textbf{Collaboration \& Version Control:} Git, GitHub, JIRA \end{onecolentry}

        \vspace{0.2 cm}

        \begin{onecolentry}
            \textbf{Cloud Platforms:} AWS, DigitalOcean \end{onecolentry}

    \section{Experience}

        \begin{twocolentry}{
            May 2024 – Aug 2024
        }
            \textbf{Summer intern}, HPCC Systems -- Alpharetta, GA\end{twocolentry}

        \vspace{0.10 cm}
        \begin{onecolentry}
            \begin{highlights}
                \item Implemented GitHub Actions Scripts to improve the continuous Integration pipelines
                \item Migrated the regression testing scripts that ran locally to GitHub Actions
                \item Designed and implemented a GitHub Actions workflow file to automatically test documentation for broken links, improving the quality of project documentation
            \end{highlights}
        \end{onecolentry}

    \section{Projects}

        \begin{twocolentry}{
        }
            \textbf{Private Cloud Cluster Setup on Raspberry Pi}\end{twocolentry}

        \vspace{0.10 cm}
        \begin{onecolentry}
            \begin{highlights}
                \item Designed and deployed a private cloud cluster using 12 Raspberry Pi nodes to build an IoT infrastructure.
                \item Set up monitoring using Prometheus, Grafana, and Node Exporter for real-time metrics and cluster health.
                \item Deployed open-source alternatives to enterprise tools including S3, Slack, Jira, and GitHub.
                \item Implemented Networking, Configured DNS resolution and set up a reverse proxy to manage traffic across services and ensure efficient routing
                \item Implemented security best practices such as VPN for remote access, a bastion for central access, etc.
            \end{highlights}
        \end{onecolentry}


        \vspace{0.2 cm}

        \begin{twocolentry}{
        }
            \textbf{GitOps CI/CD Pipeline Automation}\end{twocolentry}

        \vspace{0.10 cm}
        \begin{onecolentry}
            \begin{highlights}
                \item Implemented CI pipeline using GitHub Actions to automate testing, Docker image building, and pushing images to Docker Hub
                \item Configured Kubernetes manifests for application deployment.
                \item ArgoCD was deployed to monitor the GitHub repository for manifest changes, enabling automated Kubernetes updates.
                \item Link to my work: \href{https://github.com/Charan-Sharan/CRUD}{Development repository} | \href{https://github.com/Charan-Sharan/gitops}{GitOps repository}
            \end{highlights}
        \end{onecolentry}


        \vspace{0.2 cm}

        \begin{twocolentry}{
            
        }
            \textbf{Linux Security Hardening}\end{twocolentry}

        \vspace{0.10 cm}
        \begin{onecolentry}
            \text{Developed bash scripts to automate Linux server security configurations, that include:} 
            \begin{highlights}
                \item Creating non-root users for secure access.
                \item Changing default SSH ports.
                \item Disable root login and password-based login and replace it with key-based login.
                % \item 
            \end{highlights}
        \end{onecolentry}

\end{document}